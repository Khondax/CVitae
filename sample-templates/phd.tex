% QUEDARME LA FORMA DE PONER EL TIEMPO QUE LLEVAS CON CADA TECNOLOGÍA

%-----------------------------------------------------------------------------------------------------------------------------------------------%
%	The MIT License (MIT)
%
%	Copyright (c) 2019 Jan Küster
%
%	Permission is hereby granted, free of charge, to any person obtaining a copy
%	of this software and associated documentation files (the "Software"), to deal
%	in the Software without restriction, including without limitation the rights
%	to use, copy, modify, merge, publish, distribute, sublicense, and/or sell
%	copies of the Software, and to permit persons to whom the Software is
%	furnished to do so, subject to the following conditions:
%	
%	THE SOFTWARE IS PROVIDED "AS IS", WITHOUT WARRANTY OF ANY KIND, EXPRESS OR
%	IMPLIED, INCLUDING BUT NOT LIMITED TO THE WARRANTIES OF MERCHANTABILITY,
%	FITNESS FOR A PARTICULAR PURPOSE AND NONINFRINGEMENT. IN NO EVENT SHALL THE
%	AUTHORS OR COPYRIGHT HOLDERS BE LIABLE FOR ANY CLAIM, DAMAGES OR OTHER
%	LIABILITY, WHETHER IN AN ACTION OF CONTRACT, TORT OR OTHERWISE, ARISING FROM,
%	OUT OF OR IN CONNECTION WITH THE SOFTWARE OR THE USE OR OTHER DEALINGS IN
%	THE SOFTWARE.
%	
%
%-----------------------------------------------------------------------------------------------------------------------------------------------%


%============================================================================%
%
%	DOCUMENT DEFINITION
%
%============================================================================%

%we use article class because we want to fully customize the page and don't use a cv template
\documentclass[10pt,A4]{article}	


%----------------------------------------------------------------------------------------
%	ENCODING
%----------------------------------------------------------------------------------------

% we use utf8 since we want to build from any machine
\usepackage[utf8]{inputenc}		

%----------------------------------------------------------------------------------------
%	LOGIC
%----------------------------------------------------------------------------------------

% provides \isempty test
\usepackage{xstring, xifthen}

%----------------------------------------------------------------------------------------
%	FONT BASICS
%----------------------------------------------------------------------------------------

% some tex-live fonts - choose your own

%\usepackage[defaultsans]{droidsans}
%\usepackage[default]{comfortaa}
%\usepackage{cmbright}
\usepackage[default]{raleway}
%\usepackage{fetamont}
%\usepackage[default]{gillius}
%\usepackage[light,math]{iwona}
%\usepackage[thin]{roboto} 

% set font default
\renewcommand*\familydefault{\sfdefault} 	
\usepackage[T1]{fontenc}

% more font size definitions
\usepackage{moresize}

%----------------------------------------------------------------------------------------
%	FONT AWESOME ICONS
%---------------------------------------------------------------------------------------- 

% include the fontawesome icon set
\usepackage{fontawesome}

% use to vertically center content
% credits to: http://tex.stackexchange.com/questions/7219/how-to-vertically-center-two-images-next-to-each-other
\newcommand{\vcenteredinclude}[1]{\begingroup
\setbox0=\hbox{\includegraphics{#1}}%
\parbox{\wd0}{\box0}\endgroup}

% use to vertically center content
% credits to: http://tex.stackexchange.com/questions/7219/how-to-vertically-center-two-images-next-to-each-other
\newcommand*{\vcenteredhbox}[1]{\begingroup
\setbox0=\hbox{#1}\parbox{\wd0}{\box0}\endgroup}

% icon shortcut
\newcommand{\icon}[3] { 							
	\makebox(#2, #2){\textcolor{maincol}{\csname fa#1\endcsname}}
}	

% icon with text shortcut
\newcommand{\icontext}[4]{ 						
	\vcenteredhbox{\icon{#1}{#2}{#3}}  \hspace{2pt}  \parbox{0.9\mpwidth}{\textcolor{#4}{#3}}
}

% icon with website url
\newcommand{\iconhref}[5]{ 						
    \vcenteredhbox{\icon{#1}{#2}{#5}}  \hspace{2pt} \href{#4}{\textcolor{#5}{#3}}
}

% icon with email link
\newcommand{\iconemail}[5]{ 						
    \vcenteredhbox{\icon{#1}{#2}{#5}}  \hspace{2pt} \href{mailto:#4}{\textcolor{#5}{#3}}
}

%----------------------------------------------------------------------------------------
%	PAGE LAYOUT  DEFINITIONS
%----------------------------------------------------------------------------------------

% page outer frames (debug-only)
% \usepackage{showframe}		

% we use paracol to display breakable two columns
\usepackage{paracol}

% define page styles using geometry
\usepackage[a4paper]{geometry}

% remove all possible margins
\geometry{top=1cm, bottom=1cm, left=1cm, right=1cm}

\usepackage{fancyhdr}
\pagestyle{empty}

% space between header and content
% \setlength{\headheight}{0pt}

% indentation is zero
\setlength{\parindent}{0mm}

%----------------------------------------------------------------------------------------
%	TABLE /ARRAY DEFINITIONS
%---------------------------------------------------------------------------------------- 

% extended aligning of tabular cells
\usepackage{array}

% custom column right-align with fixed width
% use like p{size} but via x{size}
\newcolumntype{x}[1]{%
>{\raggedleft\hspace{0pt}}p{#1}}%


%----------------------------------------------------------------------------------------
%	GRAPHICS DEFINITIONS
%---------------------------------------------------------------------------------------- 

%for header image
\usepackage{graphicx}

% use this for floating figures
% \usepackage{wrapfig}
% \usepackage{float}
% \floatstyle{boxed} 
% \restylefloat{figure}

%for drawing graphics		
\usepackage{tikz}				
\usetikzlibrary{shapes, backgrounds,mindmap, trees}

%----------------------------------------------------------------------------------------
%	Color DEFINITIONS
%---------------------------------------------------------------------------------------- 
\usepackage{transparent}
\usepackage{color}

% primary color
\definecolor{maincol}{RGB}{ 45, 50, 90 }

% accent color, secondary
% \definecolor{accentcol}{RGB}{ 250, 150, 10 }

% dark color
\definecolor{darkcol}{RGB}{ 70, 70, 70 }

% light color
\definecolor{lightcol}{RGB}{245,245,245}


% Package for links, must be the last package used
\usepackage[hidelinks]{hyperref}

% returns minipage width minus two times \fboxsep
% to keep padding included in width calculations
% can also be used for other boxes / environments
\newcommand{\mpwidth}{\linewidth-\fboxsep-\fboxsep}
	


%============================================================================%
%
%	CV COMMANDS
%
%============================================================================%

%----------------------------------------------------------------------------------------
%	 CV LIST
%----------------------------------------------------------------------------------------

% renders a standard latex list but abstracts away the environment definition (begin/end)
\newcommand{\cvlist}[1] {
	\begin{itemize}{#1}\end{itemize}
}

%----------------------------------------------------------------------------------------
%	 CV TEXT
%----------------------------------------------------------------------------------------

% base class to wrap any text based stuff here. Renders like a paragraph.
% Allows complex commands to be passed, too.
% param 1: *any
\newcommand{\cvtext}[1] {
	\begin{tabular*}{1\mpwidth}{p{0.98\mpwidth}}
		\parbox{1\mpwidth}{#1}
	\end{tabular*}
}

%----------------------------------------------------------------------------------------
%	CV SECTION
%----------------------------------------------------------------------------------------

% Renders a a CV section headline with a nice underline in main color.
% param 1: section title
\newcommand{\cvsection}[1] {
	\vspace{14pt}
	\cvtext{
		\textbf{\LARGE{\textcolor{darkcol}{\uppercase{#1}}}}\\[-4pt]
		\textcolor{maincol}{ \rule{0.1\textwidth}{2pt} } \\
	}
}

%----------------------------------------------------------------------------------------
%	META SKILL
%----------------------------------------------------------------------------------------

% Renders a progress-bar to indicate a certain skill in percent.
% param 1: name of the skill / tech / etc.
% param 2: level (for example in years)
% param 3: percent, values range from 0 to 1
\newcommand{\cvskill}[3] {
	\begin{tabular*}{1\mpwidth}{p{0.72\mpwidth}  r}
 		\textcolor{black}{\textbf{#1}} & \textcolor{maincol}{#2}\\
	\end{tabular*}%
	
	\hspace{4pt}
	\begin{tikzpicture}[scale=1,rounded corners=2pt,very thin]
		\fill [lightcol] (0,0) rectangle (1\mpwidth, 0.15);
		\fill [maincol] (0,0) rectangle (#3\mpwidth, 0.15);
  	\end{tikzpicture}%
}


%----------------------------------------------------------------------------------------
%	 CV EVENT
%----------------------------------------------------------------------------------------

% Renders a table and a paragraph (cvtext) wrapped in a parbox (to ensure minimum content
% is glued together when a pagebreak appears).
% Additional Information can be passed in text or list form (or other environments).
% the work you did
% param 1: time-frame i.e. Sep 14 - Jan 15 etc.
% param 2:	 event name (job position etc.)
% param 3: Customer, Employer, Industry
% param 4: Short description
% param 5: work done (optional)
% param 6: technologies include (optional)
% param 7: achievements (optional)
\newcommand{\cvevent}[7] {
	
	% we wrap this part in a parbox, so title and description are not separated on a pagebreak
	% if you need more control on page breaks, remove the parbox
	\parbox{\mpwidth}{
		\begin{tabular*}{1\mpwidth}{p{0.72\mpwidth}  r}
	 		\textcolor{black}{\textbf{#2}} & \colorbox{maincol}{\makebox[0.25\mpwidth]{\textcolor{white}{#1}}} \\
			\textcolor{maincol}{\textbf{#3}} & \\
		\end{tabular*}\\[4pt]
	
		\ifthenelse{\isempty{#4}}{}{
			\cvtext{#4}\\
		}
	}

	\ifthenelse{\isempty{#5}}{}{
		\vspace{4pt}
		{#5}
	}
	\vspace{4pt}
}

%----------------------------------------------------------------------------------------
%	 CV META EVENT
%----------------------------------------------------------------------------------------

% Renders a CV event on the sidebar
% param 1: title
% param 2: subtitle (optional)
% param 3: customer, employer, etc,. (optional)
% param 4: info text (optional)
\newcommand{\cvmetaevent}[4] {
	\textcolor{maincol} {\cvtext{\textbf{\begin{flushleft}#1\end{flushleft}}}}

	\ifthenelse{\isempty{#2}}{}{
	\textcolor{darkcol} {\cvtext{\textbf{#2}} }
	}

	\ifthenelse{\isempty{#3}}{}{
		\cvtext{{ \textcolor{darkcol} {#3} }}\\
	}

	\cvtext{#4}\\[14pt]
}

%---------------------------------------------------------------------------------------
%	QR CODE
%----------------------------------------------------------------------------------------

% Renders a qrcode image (centered, relative to the parentwidth)
% param 1: percent width, from 0 to 1
\newcommand{\cvqrcode}[1] {
	\begin{center}
		\includegraphics[width={#1}\mpwidth]{qrcode}
	\end{center}
}

%=+=+=+=+=+=+=+=+=+=+=+=+=+=+=+=+=+=+=+=+=+=+=+=+=+=+=+=+=+=+=+=+=+=+=+=+=+=+=+=+
%,,,,,,,,,,,,,,,,,,,,,,,,,,,,,,,,,,,,,,,,,,,,,,,,,,,,,,,,,,,,,,,,,,,,,,,,,,,,,,,,
                       % EDIT AFTER THIS POINT
%''''''''''''''''''''''''''''''''''''''''''''''''''''''''''''''''''''''''''''''''
%=+=+=+=+=+=+=+=+=+=+=+=+=+=+=+=+=+=+=+=+=+=+=+=+=+=+=+=+=+=+=+=+=+=+=+=+=+=+=+=+


%============================================================================%
%
%
%
%	DOCUMENT CONTENT
%
%
%
%============================================================================%
\begin{document}
\columnratio{0.31}
\setlength{\columnsep}{2.2em}
\setlength{\columnseprule}{4pt}
\colseprulecolor{lightcol}
\begin{paracol}{2}
\begin{leftcolumn}
%---------------------------------------------------------------------------------------
%	META IMAGE
%----------------------------------------------------------------------------------------
%\includegraphics[width=\linewidth]{untitled.jpg}	%trimming relative to image size


\vfill\null
\cvsection{CONTACT}
	
\iconemail{EnvelopeSquare}{14}{yourmail@gmail.com}{yourmail@gmail.com}{black}\\[6pt]
\icontext{Github}{14}{\href{https://github.com/yourid}{yourid}}{black}\\[6pt]
\icontext{Linkedin}{14}{\href{https://www.linkedin.com/in/yourid/}{yourid}}{black}\\[6pt]
\icontext{Phone}{14}{+91 XXXXX XXXXX}{black}\\[6pt]
\vfill\null
%\cvqrcode{0.7}

%---------------------------------------------------------------------------------------
%	META SKILLS
%----------------------------------------------------------------------------------------
\cvsection{SKILLS}

%\cvskill{Skill_Name} {Years of experience} {percentage of bar fill} \\[-2pt]

\cvskill{Internet of Things} {5+ yrs} {1} \\[-2pt]

\cvskill{Machine Learning} {1+ yrs} {0.2} \\[-2pt]

\cvskill{Python} {3+ yrs} {0.6} \\[-2pt]

\cvskill{C++} {5+ yrs} {1} \\[-2pt]

\cvskill{Linux} {1+ yrs} {0.2} \\[-2pt]

\cvskill{Web Development} {5+ yrs} {1} \\[-2pt]

\cvskill{Quantum Computing} {1+ yrs} {0.2} \\[-2pt]

\cvskill{Teaching} {3+ yrs} {0.6} \\[-2pt]

%\vfill\null
%\cvqrcode{0.7}

%---------------------------------------------------------------------------------------
%	ACHIEVEMENTS
%----------------------------------------------------------------------------------------
\newpage
\cvsection{ACHIEVEMENTS}

\cvmetaevent
{GATE}
{Computer Science and Information Technology (CS)}
{}
{Qualified in 2016 with 389 score in general category.}

\cvmetaevent
{IELTS}
{7.0 out of 9 Band}
{}
{A certificate issued by International Development Program (IDP), Australia to prove English language proficiency for non-native English language speakers.}

\cvmetaevent
{Cloud Computing 101}
{94.30\%}
{}
{A certificate issued by coursera to prove basic understanding of cloud computing.}

\cvmetaevent
{Achievement}
{Certificate Detail}
{}
{Description will go here as a sentence.}

\end{leftcolumn}
\begin{rightcolumn}
%---------------------------------------------------------------------------------------
%	TITLE  HEADER
%----------------------------------------------------------------------------------------
\fcolorbox{white}{darkcol}{\begin{minipage}[c][3.5cm][c]{1\mpwidth}
	\begin {center}
		\HUGE{ \textbf{ \textcolor{white}{ \uppercase{ YOUR NAME } } } } \\[-24pt]
		\textcolor{white}{ \rule{0.1\textwidth}{1.25pt} } \\[4pt]
		\large{ \textcolor{white} {Research Scholar - Computer Science \& Engineering } }
	\end {center}
\end{minipage}} \\[14pt]
\vspace{-12pt}

%---------------------------------------------------------------------------------------
%	EDUCATION
%----------------------------------------------------------------------------------------
%\vfill\null
\cvsection{EDUCATION}

\cvevent
	{\textbf{2019 - ongoing}}
	{Ph. D. - Computer Science $\&$ Engineering}
	{IIIT - Lucknow, UP (India)}
	{Current Status: Coursework and On-campus duration completed.}
\vfill\null

\cvevent
	{\textbf{20XX - 20XX}}
	{M. Tech. - Computer Science $\&$ Engineering}
	{University Name - City, State (Country)}
	{Passed with \textbf{X.XX CGPA}. Thesis work on Your thesis domain.}
\vfill\null

\cvevent
    {\textbf{20XX - 20XX}}
    {B. Tech. - Information Technology}
    {University Name - City, State (Country)}
    {Passed with \textbf{XX.XX$\%$}. Major project was Major project name.}
\vfill\null

%---------------------------------------------------------------------------------------
%	WORK EXPERIENCE
%----------------------------------------------------------------------------------------
\vfill\null
\cvsection{WORK EXPERIENCE}

\cvevent
	{\textbf{Jan YY - Dec YY}}
	{Position name}
	{Institute/Organization, City (State)}
	{Write about your roles and responsibilities. In case of teaching experience also write about the subjects you taught.}
	{}
\vfill\null

\cvevent
	{\textbf{Jan YY - Jul YY}}
	{Position name}
	{Institute/Organization, City (State)}
	{Write about your roles and responsibilities. In case of teaching experience also write about the subjects you taught.}
	{}
\vfill\null
%---------------------------------------------------------------------------------------
%	PUBLICATION
%----------------------------------------------------------------------------------------
\vspace{-0.5cm}
\vfill\null
\cvsection{PUBLICATIONS}

\cvevent
	{\textbf{UGC Listed}}
	{Title of your research paper}
	{Journal Name (ISSN: XXXX-XXXX) Vol XX, Issue XX, 20XX}
	{Status: Accepted and Published}
	{}
\vfill\null

\cvevent
	{\textbf{SCI - IF X.XXX}}
	{Title of your research paper}
	{Journal Name (ISSN: XXXX-XXXX) Vol XX, Issue XX, 20XX}
	{Status: Under Review}
	{}
\vfill\null

\cvevent
	{\textbf{SCI - IF X.XXX}}
	{Title of your research paper}
	{Journal Name (ISSN: XXXX-XXXX) Vol XX, Issue XX, 20XX}
	{Status: Under Review}
	{}
\vfill\null
%---------------------------------------------------------------------------------------
%	PROJECTS
%----------------------------------------------------------------------------------------
\vfill\null
\cvsection{PROJECTS}

\cvevent
	{\textbf{20XX}}
	{Project Name}
	{Tool: Python, Raspberry Pi}
	{A short description of your project.}
\vfill\null


\cvevent
	{\textbf{20XX}}
	{Project Name}
	{Tool: Web Development}
	{A short description of your project.}
\vfill\null


\cvevent
	{\textbf{20XX}}
	{Project Name}
	{Tool: Android Studio}
	{A short description of your project.}
\vfill\null

%---------------------------------------------------------------------------------------
%	WORKSHOPS
%----------------------------------------------------------------------------------------
\vfill\null
\cvsection{WORKSHOPS \& CONFERENCES}

\cvevent
	{\textbf{Mon 20XX}}
	{Name of Conference}
	{Conducted by}
	{}
\vfill\null

\cvevent
	{\textbf{May 2019}}
	{IEEE Internation Conference on Future of Internet of Things}
	{IEEE \& IIT Kanpur}
	{}
\vfill\null

\cvevent
	{\textbf{Jul 2020}}
	{Online International Workshop on Machine Learning Applications to Images, IoT and Wireless Sensor Networks}
	{University of Essex and IIIT, Lucknow}
	{}
\vfill\null
%---------------------------------------------------------------------------------------
%	SKILLS
%----------------------------------------------------------------------------------------

%---------------------------------------------------------------------------------------
%	PERSONAL DETAILS
%----------------------------------------------------------------------------------------
\vfill\null
\cvsection{EXTRACURRICULAR}
\vspace{-0.3cm}
\begin{itemize}
  \item Put all the points that are not covered in \textbf{above sections}.
  \item Put all the points that are \textbf{not covered} in above sections.
  \item Put all the \textbf{points} that are not covered in above sections.
  \item \textbf{Put all the points} that are not covered in above sections.
\end{itemize}
\vfill\null


% hotfixes to create fake-space to ensure the whole height is used
\vfill
\vfill
\vfill
\end{rightcolumn}
\end{paracol}
\end{document}